
\chapter{Einleitung}

\section{Motivation}

Mit Surfen oder Wellenreiten bezeichnet man eine Wassersportart, bei
der versucht wird, auf einem Surfbrett stehend eine brechende Welle
entlang zu fahren. Anders als beim Windsurfen wird hier nicht die
Kraft des Windes, sondern die Kraft der brechenden Welle benutzt, um
die zum Aufstehen und Fahren benötigte Geschwindigkeit zu
erreichen. Zum Surfen geeignete Wellen fangen idealerweise an einem
Punkt an zu brechen, und fallen dann kontinuierlich in eine oder beide
Richtungen in sich zusammen. Der Surfer versucht die Welle kurz vor
dem brechenden Punkt anzupaddeln und aufzustehen um dann auf der Welle
so lange wie möglich in die brechende Richtung zu fahren.

Surfbare Wellen sind allerdings nicht immer dort anzutreffen wo es ein
Meer oder einen Strand gibt. Vielmehr sind für Surfer interessante
Wellen an den Orten zu finden, im folgenden \textit{Spots} genannt,
deren geografische Lage das Eintreffen von Wellen begünstigt, die weit
entfernt entstandenen und viele Kilometer weit gereist sind. Diese Art
von Wellen wird Swell, oder auch Dünung genannt und ist eine der
Grundvoraussetzung für gute Surfbedingungen.

Die Beschaffenheit des Untergrunds, über dem Wellen anfangen zu
brechen, ist ein weiterer wichtiger Faktor, der die Qualität der zu
surfenden Wellen beeinflusst. An den bei Surfern sehr beliebten
\textit{Pointbreaks} brechen die Wellen immer an der gleichen
Stelle. Hier treffen die Wellen meist auf ein Riff oder einen aus
Steinen bzw. Felsen bestehenden Untergrund, der sie abbremst und zum
Brechen bringt. Dies sind die beständigsten, am besten einschätzbaren,
aber auch gefährlichsten Spots. Besteht der Untergrund aus Sand, sind
meist sich durch Gezeiten, Strömungen und Stürme ständig verändernde
Sandbänke für das Brechen der Wellen verantwortlich. Diese Spots sind
weniger beständig und verändern sich im Laufe der Zeit sehr viel
schneller als die beständigeren \textit{Pointbreaks}.

\begin{figure}[h]
  \includegraphics[width=\textwidth]{bilder/intro}
\end{figure}

Weitere wichtige Faktoren, die sich auf die Eigenschaften von
surfbaren Wellen auswirken, sind die Gezeiten, die Richtung aus der
die Wellen kommen, sowie die Wind- und Wetterverhältnisse in den
jeweiligen Jahreszeiten. Insbesondere die Windstärke und die
Windrichtung sind hier von großem Interesse. Surfbare Wellen können
sehr schnell von zu starkem Wind aus der falschen Richtung zunichte
gemacht werden. Weiterhin ist die Ausrichtung der Spots von Interesse,
da einige sehr viel windgeschützter sind als andere.

Da surfbare Wellen von vielen Faktoren beeinflusst werden beschäftigen
sich die meisten Surfer deshalb vor und während ihren Reisen
insbesondere mit den aktuellen Wetter- und Wellenvorhersagen und
versuchen herauszufinden welche Spots bei welchen Verhältnissen am
besten brechen. Der typische Surfer liegt deshalb nicht nur faul am
Strand herum und wartet dort auf Wellen, sondern ist oft auf
abgelegenen Straßen und Trampelpfaden entlang der Küste unterwegs, in
der Hoffnung einen abgelegenen Spot mit den perfekten Wellen zu
finden.

\section{The Stormrider Guides}
Die sogenannten \textit{Stormrider Guides} des \textit{Low Pressure}
\footnote{\url{http://www.lowpressure.co.uk}} Verlags sind seit langem
die populärsten Reiseführer in der Surfszene. Sie erfreuen sich Dank
der vielen hilfreichen Informationen und Tipps rund ums Surfen einer
sehr großen Beliebtheit. Die nach Kontinenten und Ländern gegliederten
Bücher enthalten Reiseinformationen über Land und Leute, Kultur, Klima
sowie Kartenausschnitte mit Beschreibungen zu den Surfbedingungen an
den Spots. Insbesondere die detaillierten Informationen über die
Eigenschaften der Wellen und der Umgebung sind von großem
Nutzen. Beispielsweise wird beschrieben zu welcher Gezeit
bzw. \textit{Tide} die Wellen an einem Spot am besten brechen, wie
stark die Meeresströmung ist, ob die Brandung an einem Sandstrand oder
auf einem flachen Riff ist oder ob irgendwelche Gefahren zu beachten
sind. 

\begin{figure}[h]
  \includegraphics[width=\textwidth]{bilder/mavericks}
  \caption{\textit{''Don't even think about riding big Mavericks''} -
    Hinweis des \textit{World Stormrider Guide} zu \textit{Mavericks},
    einem der bekanntesten \textit{Big Wave} Spots in Nordkalifornien,
    USA.}
\end{figure}

\section{Swell - Die Entstehung von Wellen}
Die besten Surfspots auf der Welt sind meist in den Ländern zu finden,
in denen regelmäßig \textit{Swell} eintrifft. Mit \textit{Swell} oder
\textit{Dünung} werden Seewellen bezeichnet, deren Entstehungs\-gebiet
weit entfernt ist von dem Ort an dem sie eintreffen und brechen. Swell
wird von den unterschiedlichsten Wetter\-phänomenen erzeugt, zu denen
z.B. Wirbelstürmen, Passatwinden, Monsune und Tiefdruckgebiete gehören
\cite[S.15]{storm_europe_1998}. Der in Europa eintreffender Swell
entsteht dabei hauptsächlich durch die Luftzirkulation in
Tiefdruckgebieten über dem Atlantik. Bei ruhigen Wetterverhältnissen
kann man eintreffenden Swell sehr gut erkennen, da die eintreffenden
Wellen mit einigem Abstand linienförmig angeordnet sind. Diese saubere
Anordnung ist besonders gut in Abbildung \ref{swell-lines} zu
erkennen.

\begin{figure}[h]
 \includegraphics[width=\textwidth]{bilder/swell}
 \caption{Linienförmig eintreffender Swell}
 \label{swell-lines}
\end{figure}

Am Ursprungsort entstehen Wellen dadurch, dass die Wasseroberfläche
durch die über ihr strömenden Winde in Bewegung versetzt wird. Die
Gipfel und Täler der Wellen werden durch die Zirkulation des Windes
über der Oberfläche immer höher und tiefer, bis sie ein Limit
erreichen und in sich zusammenbrechen. Die Höhe der Wellen ist dabei
von der Stärke, der Dauer und der Strecke über die der Wind strömt
abhängig. Die Wellen breiten sich vom Entstehungsort kreisförmig auf
ihre Umgebung im Meer aus. Die Geschwindigkeit der Ausbreitung hängt
dabei von dem Abstand zwischen zwei Wellengipfeln ab, der sogenannten
Wellenlänge. Das Wellenchaos am Entstehungsort mit vielen
unterschiedlichen Wellenlängen beginnt sich mit der Ausbreitung
langsam zu legen. Die schnelleren Wellen, mit weiter auseinander
liegenden Gipfeln, beginnen die langsameren Wellen zu überholen und
fangen an sich linienförmig zu ordnen. Die vorderen Wellen werden
dabei zu den kräftiger und sauber angeordneten, die hinteren Wellen zu
den schwächeren und chaotischeren. Je weiter die Strecke die ein Swell
hinter sich gelegt hat und linienförmiger er geordnet ist, desto
größer ist die Geschwindigkeit und die Kraft der Wellen am dem Ort an
dem sie brechen. Das Eintreffen von Swell ist eine der
Grundvoraussetzungen für gute und surfbare Wellen. Deshalb gibt es
z.B. auch in Deutschland keine guten Wellen, da der potentiell
eintreffende Swell meist durch England abgeschirmt wird.

\section{Ziel der Diplomarbeit}
Ziel der Diplomarbeit ist die Entwicklung einer \textit{Community
  Plattform} für Surfer, welche den Grundgedanken der
\textit{Stormrider Guides} aufgreift und mit den neuen Möglichkeiten
des Internets und des \textit{Web 2.0} verknüpft. Grundlage der
Plattform sollen Informationen zu den Surf Bedingungen an den
verschiedenen Spots sein, die gemeinschaftlich durch die Mitglieder
der Community erstellt und bearbeitet werden können. Diese Spot
Beschreibungen sollen mit aktuellen Wetter- und Wellenvorhersagen
verknüpft und den Besuchern der Plattform zur Verfügung gestellt
werden. Durch die Integration externer Dienste wie \textit{Google
  Maps}, \textit{Flickr} und \textit{YouTube} sollen die Informationen
aufgewertet und dem Nutzer ein besserer Überblick verschafft
werden. In dieser Arbeit werden einige der dabei aufgetretenen
Probleme und Lösungen vorgestellt und diskutiert. Im Kapitel
\textit{Anforderungen an die Web Applikation} wird auf die
Funktionalität der Anwendung und auf einige der zu Grunde liegenden
Konzepte und Dienste eingegangen. Die zur Umsetzung verwendeten
Technologien und einige nennenswerte Methoden zur Entwicklung von Web
Applikationen werden im Kapitel \textit{Implementierung der Web
  Applikation} vorgestellt. Der Hauptteil dieser Arbeit beschäftigt
sich mit der Integration der Wetter- und Wellenvorhersagen, die im
Kapitel \textit{Aufbau und Analyse der ETL Prozesse} näher beschrieben
werden. Im letzten Kapitel werden schließlich einige Vorschläge
gemacht, mit denen die Wetter- und Wellenvorhersagen verbessert werden
könnten, indem die lokalen Gegebenheiten eines Spots indirekt mit
einbezogen werden. Einer dieser Vorschläge bedient sich dabei
Verfahren aus dem Bereich des Data Mining, ist von sehr
experimenteller Natur und konnte hier aus Mangel an Trainingsdaten und
Ortskenntnissen leider nicht weiter verfolgt werden.

%%% Local Variables:
%%% mode: latex
%%% TeX-master: "../community-plattform"
%%% End:
