\documentclass[12pt,a4paper,ngerman]{report}
\usepackage{babel}
\usepackage{graphicx}
\usepackage[utf8]{inputenc}

%\usepackage{float}
\usepackage{floatflt}

\author{Roman Scherer \\ scherer@informatik.hu-berlin.de}

\title{{\small
    Humboldt-Universität zu Berlin \\
    Mathematisch-Naturwissenschaftliche Fakultät II \\
    Institut für Informatik \\
    Lehrstuhl für Systemarchitektur \\
    Unter den Linden 6 \\
    D-10099 Berlin \\
  } \vspace{0.4cm}
  \includegraphics[width=4cm]{bilder/husiegel_bw} \\
  \vspace{0.4cm}
  \large{Diplomarbeit} \\
  \LARGE{\textbf{Konzeption und Implementierung einer Community-Plattform für Surfer}} \\
}


\begin{document}

\maketitle

%
\begin{abstract}
  Surfen oder Wellenreiten ist eine Wassersportart, die besonders
  stark vom Wetter, den Gezeiten und den Eigenschaften der zu
  surfenden Wellen abhängig ist. Im Rahmen dieser Diplomarbeit soll
  eine \textit{Community-Plattform} für das Internet entwickelt
  werden, auf der Surfer für sie wichtige Informationen finden und
  austauschen können. Die bei der Konzeption und Implementierung
  auftretenden Probleme und Lösungen sollen in dieser Arbeit
  diskutiert werden.
\end{abstract}

%%% Local Variables:
%%% mode: latex
%%% TeX-master: "./community-plattform"
%%% End:


\tableofcontents

%
\section{Einleitung}

\subsection{Motivation}

Mit Surfen oder Wellenreiten bezeichnet man eine Wassersportart, bei
der versucht wird, stehend eine brechende Welle mit Hilfe eines
Surfbretts entlang zu fahren. Anders als beim Windsurfen wird hier
nicht die Kraft des Windes, sondern die Kraft der brechenden Welle
benutzt, um die zum Aufstehen und Fahren benötigte Geschwindigkeit zu
erreichen. Zum Surfen geeignete Wellen brechen idealerweise
kontinuierlich in eine Richtung.

Diese Wellen sind allerdings nicht immer dort anzutreffen wo es ein
Meer oder einen Strand gibt. Vielmehr sind für Surfer interessante
Wellen an den Orten zu finden, im folgenden \textit{Spots} genannt,
deren geografische Lage die Entstehung von surfbaren Wellen
begünstigt.

Die Beschaffenheit des Untergrunds, über dem die Wellen brechen, ist
dabei einer der wichtigsten Faktoren. Die bei Surfern sehr beliebten
\textit{Pointbreaks} brechen meistens an einer bestimmten Stelle über
festem Untergrund. Diese sind die beständigsten, am besten
einschätzbaren, aber auch gefährlichsten Spots, denn hier brechen die
Wellen meistens über Riffen oder steinigem Boden. Besteht der
Untergrund aus Sand, sind durch Gezeiten, Strömungen und Stürme sich
ständig verändernde Sandbänke für das Brechen der Wellen
verantwortlich. Weitere wichtige Faktoren, die sich auf die
Eigenschaften von surfbaren Wellen auswirken, sind die Gezeiten, die
Richtung aus der die Wellen kommen, sowie die Wind- und
Wetterverhältnisse in den jeweiligen Jahreszeiten.

Die sogenannten \textit{Stormrider Guides} des \textit{Low Pressure}
Verlags sind seit langem die populärsten Reiseführer in der
Surfszene. Sie erfreuen sich dank der vielen hilfreichen Informationen
und Tipps einer sehr großen Beliebtheit. Die nach Kontinenten und
Ländern gegliederten Bücher enthalten Reiseinformationen über Land und
Leute, Kultur, Klima sowie Kartenausschnitte mit Beschreibungen zu den
Surfbedingungen, die an den jeweiligen Spots herrschen. Insbesondere
die detaillierten Informationen über die Eigenschaften der Wellen und
der Umgebung sind von großem Nutzen. Zum Beispiel wird beschrieben zu
welcher Gezeit bzw. Tide die Wellen an einem Spot am besten brechen,
wie stark die Meeresströmung ist oder ob die Brandung an einem
Sandstrand oder auf einem flachen Riff ist.

% 
{\sf \footnotesize
  \begin{tabular}{|p{2.5cm}p{0.7cm}p{0.7cm}|}
    \hline
    & & \\
    \multicolumn{3}{|l|}{\textbf{Hai Angriff Statistik}} \\
    \multicolumn{3}{|l|}{für Länder mit mehr als 10 Angriffen} \\
    & & \\
    & \textbf{Tota}l & \textbf{Fatal} \\
    \textbf{Europa} & \textbf{38} & \textbf{18} \\
    Italien & 14 & 4 \\
    \textbf{Afrika} & \textbf{255} & \textbf{67} \\
    Südafrika & 208 & 41 \\
    Mosambik & 11 & 3 \\
    \textbf{Indischer Ozean} & \textbf{62} & \textbf{27} \\
    Maskarenen & 21 & 12 \\
    Iran & 23 & 8 \\
    Indien & 10 & 4 \\
    \textbf{Ost Asien} & \textbf{89} & \textbf{38} \\
    Papua-Neuguinea & 36 & 15 \\
    Philippinen & 15 & 6 \\
    Japan & 19 & 12 \\
    \textbf{Australien \& Neuseeland} & \textbf{326} & \textbf{141} \\
    West Australien & 28 & 9 \\
    Süd Australien & 30 & 16 \\
    Victoria & 20 & 8 \\
    Tasmanien & 16 & 6 \\
    Neusüdwales & 123 & 62 \\
    Queensland & 101 & 47 \\
    \textbf{Pazifik} & \textbf{211} & \textbf{62} \\
    Marshallinseln & 12 & 0 \\
    Salomon-Inseln & 17 & 8 \\
    Fiji & 25 & 10 \\
    Hawaii & 104 & 19 \\
    \textbf{Nord Amerika} & \textbf{720} & \textbf{38} \\
    Oregon & 17 & 1 \\
    Kalifornien & 111 & 8 \\
    Texas & 30 & 3 \\
    Florida & 187 & 13 \\
    Südkarolina & 43 & 3 \\
    Nordkarolina & 24 & 3 \\
    New Jersey & 16 & 5 \\
    \textbf{Zentralamerika} & \textbf{118} & \textbf{50} \\
    Mexiko & 39 & 21 \\
    Panama & 16 & 9 \\
    \textbf{Südamerika} & \textbf{89} & \textbf{21} \\
    Brasilien & 81 & 20 \\
    & & \\
    \textbf{TOTAL} & \textbf{1909} & \textbf{456} \\
    & & \\
    \multicolumn{3}{|l|}{The International Shark Attack File,} \\
    \multicolumn{3}{|l|}{Florida Museum of Natural History,} \\
    \multicolumn{3}{|l|}{University of Florida.} \\
%    \multicolumn{3}{|l|}{{\tiny http://www.flmnh.ufl.edu/fish/Sharks/ISAF/ISAF.htm}} \\
    \hline
  \end{tabular}
}

%%% Local Variables:
%%% mode: latex
%%% TeX-master: "../community-plattform"
%%% End:


%%% Local Variables:
%%% mode: latex
%%% TeX-master: "../community-plattform"
%%% End:

%
\chapter{Anforderungen an die Web Applikation}

\section{Spot Beschreibungen}

Eine der bekanntesten Wellen in Europa ist im spanischen Baskenland an
einer Fluss\-mündung östlich des Ortes \textit{Mundaka} zu finden. Im
\textit{Stormrider Guide Europe} \cite[S.180]{storm_europe_1998}
werden die Surfbedingungen an diesem Spot wie folgt beschrieben.

\textit{Some of the longest, hollowest lefts in Europe break over
  sandbanks at the mouth of the river Gernike. It's best at low tide,
  when the rivers current (a useful conveyor belt) is less intense,
  and holds swell up to 4m (12ft). On the other side of the rivermouth
  are the beaches Laida and Laga where you can find waves at small
  swells. The river and its estuary are a Worldwide Fund for Nature
  reserve, but even so, the water is not as clean as could expected.
}

Zusätzlich werden den Beschreibungen Piktogramme aus verschiedenen
Kategorien zugeordnet, welche die Gegebenheiten vor Ort durch eine
vereinfachte grafische Darstellung widerspiegeln. Einige dieser
Kategorien sind die bevorzugte Windrichtung, die optimale Gezeit, die
Art der brechenden Welle, Beschaffenheit des Bodens, sowie mögliche
Gefahren. Kennt man die Piktogramme, ist durch einen kurzen Blick
schnell ersichtlich welche Bedingungen an einem Spot herrschen.

\begin{figure}[h]
  \begin{center}
    \includegraphics[height=40px]{bilder/mundaka-conditions}
    \caption{Piktogramme zu den Surfbedingungen in Mundaka}
    \label{piktogramm}
  \end{center}
\end{figure}

In Abbildung \ref{piktogramm} sind die Piktogramme zu sehen, die für
die obige Beschreibung in Mundaka verwendet wurden. Sie sollen
vermitteln, dass dieser Spot sehr bekannt ist und an guten Tagen mit
vielen Surfern zu rechnen ist. Die Welle bricht mit viel Kraft von
links nach rechts (\textit{Left-hander}, immer vom Strand aus gesehen)
über einer Sandbank, wobei vereinzelt Surfbretter zu Bruch gehen
können.  Sie ist nicht bei Flut (\textit{High Tide}) surfbar, und
ablandiger Wind (\textit{Offshore}) aus dem Süden trägt dazu bei, dass
die Wellen geglättet werden, später brechen und hohler werden.

Das Konzept der Stormrider Guides soll die Grundlage der Web
Applikation bilden und mit den neuen Möglichkeiten des Internets
verknüpft werden. Die Beschreibungen zu den Spots und deren
Gegebenheiten sollen gemeinschaflich durch die Mitglieder der Surf
Community verfasst werden, und als sogenannter \textit{User Generated
  Content} verwaltet werden. Die gemeinsam erstellten Beschreibungen
sollen eine objektive Sicht auf die Gegebenheiten und Surfbedingungen
an den jeweiligen Spots bieten. Für persönliche Ansichten und
Diskussionen sollen zusätzlich Kommentarfunktionen zur Verfügung
gestellt werden.

Um Spam und anderer mutwilliger Zerstörung oder Verunreinigung des
Contents vorzubeugen ist eine Registrierung der Nutzer
erforderlich. Für alle Informationen die zu einem Spot gehören und von
Nutzern der Web Applikation verändert werden können, soll eine
Historie verwaltet werden. Dies soll sicherstellen, dass bei einer
eventuellen Veruneinigung des Contents auf eine frührere Version der
Information zurückgegriffen, und diese wiederhergestellt werden kann.

\section{Kartenmaterial}

Um das Auffinden von Spots zu vereinfachen sollen diese auf einer
Karte dargestellt werden. Webservice Dienste wie \textit{Google Maps},
\textit{Yahoo! Maps} oder Microsoft's \textit{Bing Maps}
\footnote{seit Juni 2009, früher: Microsoft's Virtual Earth} bieten
die Möglichkeit interaktive Karten per Java\-script oder Flash in eine
Webseite einzubetten. Diese Dienste bieten nicht nur die typischen
Land- bzw. Straßenkarten an, sondern stellen auch Satelitenbilder und
teilweise 3-dimensionale Ansichten für bestimmte Gebiete zur
Verfügung. In Abbildung \ref{google-maps} ist eine Karten- und eine
Satellitenansicht von dem Surf Spot in \textit{Mundaka} zu
sehen. Insbesondere die Satellitenansicht ist beim Auffinden von neuen
oder geheimen Surf Spots recht nützlich, da man darauf sehr gut
brechende Wellen erkennen und sich einen Überblick auf das Gelände
verschaffen kann.

\begin{figure}[h]
  \subfigure[Karten Ansicht]{\includegraphics[width=0.49\textwidth]{bilder/google-maps-map}}
  \subfigure[Satelliten Ansicht]{\includegraphics[width=0.49\textwidth]{bilder/google-maps-hybrid}}
  \caption{Google Maps Kartenmaterial für Mundaka, Spanien}
  \label{google-maps}
\end{figure}

\section{Wetter- und Wellendaten}
\label{sec:Wetter- und Wellendaten}

Wie schon in der Einleitung erwähnt ist das Vorhandensein von Swell
eine der Grundvoraussetzungen zum Surfen. Viele Surfer nutzen deshalb
regelmäßig Dienste im Internet um sich über die Wetter- und
Wellenverhältnisse in den nächsten Tagen zu informieren. Dabei ist
hauptsächlich die Wellenhöhe, die Wellenperiode und die Stärke und
Richtung des Windes von Interesse. Die Spot Beschreibungen sollen
deshalb mit aktuellen Wetter- und Wellenvorhersagen verknüpft werden,
um den Benutzern der Web Applikation einen Mehrwert zu bieten. Die
\textit{National Oceanic and Atmospheric Administration (NOAA)} ist
die Wetter- und Ozeanografiebehörde der Vereinigten Staaten. Sie
besteht aus 5 größeren Organisationen, zu denen unter anderen auch der
\textit{National Weather Service} und der \textit{National Ocean
  Service} gehören, welche die benötigten Wetter- und Wellendaten zur
Verfügung stellen. Viele der im Internet verfügbaren Dienste, die
Wetter- oder Wellenvorhersagen anbieten, beziehen ihre Daten ebenfalls
von diesen Organisationen. Diese Vorhersagen sind zwar nicht sehr
genau, tragen aber einen wichtigen Teil dazu bei die Surfbedingungen
in einer bestimmten Region oder an einem Spot besser einzuschätzen zu
können.

\section{Bilder \& Videos}

Um Surfern einen visuellen Eindruck von einem Spot zu bieten, sollen
die Spots mit Bildern und Videos verknüpft und somit aufgewertet
werden. Auf Internetseiten wie \textit{Flickr} und \textit{YouTube}
sind von vielen bekannten Spots Bilder und Videos zu finden. Eine
Suchanfrage nach einem der bekanntesten Spots mit den Stichwörtern
\textit{Mavericks} und \textit{Surf} ergab im Juni 2009 bei
\textit{Flickr} 2319 und bei \textit{YouTube} 548 Ergebnisse. Diese in
der Surf Community gerne gesehenen Bilder und Videos sind ideal um das
bisher aus Beschreibungen und Wetter- und Wellenvorhersagen bestehende
Informationsangebot zu erweitern. Sowohl \textit{Flickr} als auch
\textit{YouTube} bieten eine Webservice Schnittstelle an, mit der es
möglich ist deren Bilder und Videos in eigene Anwendungen zu
integrieren. Zudem soll den Nutzern die Möglichkeit gegeben werden
ihre eigenen Bilder und Videos auf die Community Plattform zu laden
und dort zu veröffentlichen.

\begin{figure}[h]
  \begin{center}
    \includegraphics[width=\textwidth]{bilder/photos-flickr}
    \caption{Integration von Bildern durch den Flickr Webservice}
    \label{piktogramm}
  \end{center}
\end{figure}

\begin{figure}[h]
  \begin{center}
    \includegraphics[width=\textwidth]{bilder/videos-youtube}
    \caption{Integration von Videos durch den YouTube Webservice}
    \label{piktogramm}
  \end{center}
\end{figure}

\section{Community Funktionen}

Einige Funktionen, die aus Community Plattformen wie \textit{Facebook}
und \textit{StudiVZ} bekannt sind sollen auch in dieser Web
Applikation zu Verfügung stehen. Ziel ist auch hier einen Mehrwert für
die Nutzer zu generieren, um diese näher an die Plattform zu binden.

\subsection{Freundschaften}
Nutzer sollen in der Lage sein Freundschaftsanfragen an andere Nutzer
zu stellen, und Anfragen anderer Nutzern zu akzeptieren
bzw. abzulehnen.

\subsection{Nachrichten}

\subsection{Kommentare}
\label{subsec:Kommentare}

Spot Beschreibungen sollen eine objektive Sicht auf die


\subsection{Secret Spots}
Ein relativ unschönes aber teilweise auch verständliches Phänomen in
der Surf Szene ist der sogenannte Lokalismus oder \textit{Localism} im
Englischen. Entgegen dem allgemeinen Klischee, dass Surfer friedliche
und entspannte Menschen sind, wird dieser Eindruck an einigen Spots
durch unsportliches Verhalten einiger weniger getrübt. Beim Surfen
bekommt meist derjenige die Welle, der sie am schnellsten anpaddelt
und die bessere Technik antrainiert hat. Bei guten Surfbedingungen mit
vielen Leuten führt dies unweigerlich dazu, dass einige Surfer die
Wellen fast immer und andere fast nie bekommen.

\begin{figure}[h]
  \begin{center}
    \includegraphics[width=\textwidth]{bilder/locals-only}
    \caption{\textit{''Only Rodiles Locals''} - Lokalismus in Rodiles,
      Spanien}
    \label{locals-only}
  \end{center}
\end{figure}

Insbesondere an Wochenenden und Feiertagen, an denen viele Surfer
ihrem Hobby nachgehen, sind viele Spots überfüllt und es kann zu
aggressivem Verhalten auf dem Wasser kommen. Vor Ort ansässige Surfer,
sogenannte \textit{Locals}, fühlen sich durch die vielen Besucher an
''ihren'' Spots bedrängt und reagieren auf diese teilweise sehr
aggressiv. Wie in Abbildung \ref{locals-only} zu sehen ist wird
fremden Surfern deshalb oft durch Grafittis oder ähnlichen
Markierungen nahe gelegt, dass sie an einem Spot nichts zu suchen
haben. Dies gelingt teilweise auch, da es zuweilen vorkommt, dass man
einen Spots zwar nicht ängstlich aber zumindest mit einem unguten
Gefühl besucht. Grundsätzlich gilt je weniger Surfer an einem Spot
sind, desto entspannter ist die Atmosphäre und desto besser kommt
jeder auf seine Kosten. Abgelegenere Spots werden deshalb auch oft als
geheim gehandelt, gegenüber Fremden nicht erwähnt oder diese sogar in
die Irre geführt.

Um die vor Ort ansässigen Surfer zu respektieren und nicht sofort
jeden neuen Spot in die Welt hinaus zu posaunen wurde hier das Konzept
der sogenannten \textit{Secret Spots} eingeführt. Auf der Plattform
sollen nur die allgemein bekannten Spots für jeden sichtbar
sein. Diese Spots befinden sich meist an bekannten Stränden, sind
ausgeschildert und auch in den Karten vieler Tourismusbüros
verzeichnet. Andere Spots können beim Erstellen als \textit{Secret
  Spots} markiert werden und sind dann nur für den Ersteller selbst
und seine Freunde sichtbar. Alle anderen Funktionen, wie Wetter- und
Wellenvorhersagen stehen für diese Spots ebenfalls zur Verfügung,
allerdings nur für einen eingeschränkten Nutzerkreis.

%%% Local Variables:
%%% mode: latex
%%% TeX-master: "../community-plattform"
%%% End:


\chapter{Einleitung}
\section{Motivation}
\section{Die Entstehung von Wellen}
\section{Voraussetungen zum Surfen}
\section{Ziel der Diplomarbeit}

\chapter{Anforderungen an die Web Applikation}
\section{Wetter- und Wellendaten}
\section{Kartenmaterial}
\section{Community Funktionen}

\chapter{Implementierung der Web Applikation}
\section{Vorstellung der verwendeten Technologien}
\section{Behavior Driven Development}
\section{Architektur der Web Applikation}
\section{Caching Verfahren in Ruby on Rails}

\chapter{Aufbau und Analyse der ETL Prozesse}
\section{Überblick der verwendeten Datenquellen}
\subsection{Wetterdaten}
\subsection{Photos \& Videos}
\section{Datenbank Design}
\subsection{Konzeptionelle Schema}
\subsection{Physisches Schema}
\section{Extraktion aus den Quellsystemen}
\section{Transformation der Daten}
\section{Laden der Daten}
\section{Verbesserungen}

\chapter{Ausblick}
\section{Visualisierung von Wetter- und Wellendaten}
\section{Data Mining Verfahren}

\bibliographystyle{geralpha}
\bibliography{literatur}

\end{document}