
\chapter{Anforderungen an die Web Applikation}

\section{Spot Beschreibungen}

Eine der bekanntesten Wellen in Europa ist im spanischen Baskenland an
einer Fluss\-mündung östlich des Ortes \textit{Mundaka} zu finden. Im
\textit{Stormrider Guide Europe} werden die Surfbedingungen an diesem
Spot wie folgt beschrieben.

\textit{Some of the longest, hollowest lefts in Europe break over
  sandbanks at the mouth of the river Gernike. It's best at low tide,
  when the rivers current (a useful conveyor belt) is less intense,
  and holds swell up to 4m (12ft). On the other side of the rivermouth
  are the beaches Laida and Laga where you can find waves at small
  swells. The river and its estuary are a Worldwide Fund for Nature
  reserve, but even so, the water is not as clean as could expected.
} \cite[S.180]{storm_europe_1998}

Zusätzlich werden den Beschreibungen Piktogramme aus verschiedenen
Kategorien zugeordnet, welche die Gegebenheiten vor Ort durch eine
vereinfachte grafische Darstellung widerspiegeln. Einige dieser
Kategorien sind die bevorzugte Windrichtung, die optimale Gezeit, die
Art der brechenden Welle, Beschaffenheit des Bodens, sowie mögliche
Gefahren. Kennt man einmal die Piktogramme, ist durch einen kurzen
Blick schnell ersichtlich welche Bedingungen an einem Spot herrschen.

\begin{figure}[h]
  \begin{center}
    \includegraphics[height=40px]{bilder/mundaka-conditions}
    \caption{Piktogramme der Surf Bedingungen in Mundaka}
    \label{piktogramm}
  \end{center}
\end{figure}

In Abbildung \ref{piktogramm} sind die Piktogramme zu sehen, die für
die obige Beschreibung in Mundaka verwendet wurden. Sie sollen
vermitteln, dass dieser Spot sehr bekannt ist und an guten Tagen mit
vielen anderen Surfern zu rechnen ist. Die Welle bricht mit viel Kraft
von links nach rechts (\textit{Left-hander}, immer vom Strand aus
gesehen) über einer Sandbank, wobei vereinzelt Surfbretter zu Bruch
gehen können.  Sie ist nicht bei Flut (\textit{High Tide}) surfbar,
und ablandiger Wind (\textit{Offshore}) aus dem Süden trägt dazu bei,
dass die Wellen geglättet werden, später brechen und hohler werden.

Das Konzept der Stormrider Guides soll die Grundlage der Web
Applikation bilden und mit den Möglichkeiten des Internets verknüpft
werden. Die Beschreibungen zu den Spots sollen gemeinschaflich durch
die Mitglieder der Surf Community erstellt werden, und als sogenannter
\textit{User Generated Content} verwaltet werden.


\section{Kartenmaterial}

Um das Auffinden von Spots zu vereinfachen sollen diese auf einer
Karte dargestellt werden. Webservice Dienste wie \textit{Google Maps},
\textit{Yahoo! Maps} oder Microsoft's \textit{Bing Maps}
\footnote{seit Juni 2009, früher: Microsoft's Virtual Earth} bieten
die Möglichkeit interaktive Karten per Java\-script oder Flash in eine
Webseite einzubetten. Diese Dienste bieten nicht nur die typischen
Land- bzw. Straßenkarten an, sondern stellen auch Satelitenbilder und
teilweise 3-dimensionale Ansichten für bestimmte Gebiete zur
Verfügung.

\section{Wetter- und Wellendaten}

\section{Community Funktionen}

%%% Local Variables:
%%% mode: latex
%%% TeX-master: "../community-plattform"
%%% End:
