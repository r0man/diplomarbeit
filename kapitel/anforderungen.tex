
\chapter{Anforderungen an die Web Applikation}

\section{Spot Beschreibungen}

Eine der bekanntesten Wellen in Europa ist im spanischen Baskenland an
einer Fluss\-mündung östlich des Ortes \textit{Mundaka} zu finden. Im
\textit{Stormrider Guide Europe} werden die Surfbedingungen an diesem
Spot wie folgt beschrieben.

\textit{Some of the longest, hollowest lefts in Europe break over
  sandbanks at the mouth of the river Gernike. It's best at low tide,
  when the rivers current (a useful conveyor belt) is less intense,
  and holds swell up to 4m (12ft). On the other side of the rivermouth
  are the beaches Laida and Laga where you can find waves at small
  swells. The river and its estuary are a Worldwide Fund for Nature
  reserve, but even so, the water is not as clean as could expected.
} \cite[S.180]{storm_europe_1998}

Grundlage der Web Applikation soll ein Wiki \footnote{hawaiisch für
  schnell} System sein, das den Mitgliedern der Surf Community die
Möglichkeit gibt gemeinschaftlich Spot Beschreibungen zu erstellen und
zu bearbeiten.

\begin{figure}[h]
  \begin{center}
    \includegraphics[height=40px]{bilder/mundaka-conditions}
    \caption{Piktogramme der Surf Bedingungen in Mundaka}
    \label{piktogramm}
  \end{center}
\end{figure}

Zusätzlich zu den textuellen Beschreibungen werden in den Stormrider
Guides Piktogramme verwendet, welche die Gegebenheiten vor Ort durch
eine vereinfachte grafische Darstellung widerspiegeln. In Abbildung
\ref{piktogramm} sind die Piktogramme zu sehen, die für die obige
Beschreibung in Mundaka verwendet wurden. Sie sollen vermitteln, dass
dieser Spot sehr bekannt ist und an guten Tagen mit vielen anderen
Surfern zu rechnen ist. Die Welle bricht mit viel Kraft von links nach
rechts (\textit{Left-hander}) über einer Sandbank, wobei vereinzelt
Surfbretter zu Bruch gehen können.  Sie ist nicht bei Flut
(\textit{High Tide}) surfbar, und ablandiger Wind (\textit{Offshore})
aus dem Süden trägt dazu bei, dass die Wellen geglättet werden, später
brechen und somit hohler werden.

\section{Wetter- und Wellendaten}

\section{Kartenmaterial}

\section{Community Funktionen}

%%% Local Variables:
%%% mode: latex
%%% TeX-master: "../community-plattform"
%%% End:
