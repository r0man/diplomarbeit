
\section{Einleitung}

\subsection{Motivation}

Mit Surfen oder Wellenreiten bezeichnet man eine Wassersportart, bei
der versucht wird, stehend eine brechende Welle mit Hilfe eines
Surfbretts entlang zu fahren. Anders als beim Windsurfen wird hier
nicht die Kraft des Windes, sondern die Kraft der brechenden Welle
benutzt, um die zum Aufstehen und Fahren benötigte Geschwindigkeit zu
erreichen. Zum Surfen geeignete Wellen brechen idealerweise
kontinuierlich in eine Richtung.

Diese Wellen sind allerdings nicht immer dort anzutreffen wo es ein
Meer oder einen Strand gibt. Vielmehr sind für Surfer interessante
Wellen an den Orten zu finden, im folgenden \textit{Spots} genannt,
deren geografische Lage die Entstehung von surfbaren Wellen
begünstigt. Weitere wichtige Faktoren, die sich auf die Eigenschaften
von Wellen auswirken, sind unter anderen die Gezeiten, die
Beschaffenheit des Untergrunds, sowie Wind- und Wetterverhältnisse in
den jeweiligen Jahreszeiten.

Die sogenannten \textit{Stormrider Guides} \cite{storm_2007} des
\textit{Low Pressure} Verlags sind seit langem die populärsten
Reiseführer in der Surfszene. Sie erfreuen sich dank der vielen
hilfreichen Informationen und Tipps einer sehr großen Beliebtheit. Die
nach Kontinenten und Ländern gegliederten Bücher enthalten
Reiseinformationen über Land und Leute, Kultur, Klima sowie
Kartenausschnitte mit Beschreibungen zu den Surfbedingungen, die an
den jeweiligen Spots herrschen. Insbesondere die detaillierten
Informationen über die Eigenschaften der Wellen und der Umgebung sind
von großem Nutzen. Zum Beispiel wird beschrieben zu welcher Gezeit
bzw. Tide die Wellen an einem Spot am besten brechen, wie stark die
Meeresströmung ist oder ob die Brandung an einem Sandstrand oder auf
einem flachen Riff ist.

%%% Local Variables:
%%% mode: latex
%%% TeX-master: "../community-plattform"
%%% End:
